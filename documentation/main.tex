\documentclass{article}
\usepackage[utf8]{inputenc}
\usepackage{amsmath}
\usepackage{hyperref}
\usepackage{array}
\allowdisplaybreaks
\title{Necklace}
\author{Maksymilian Demitraszek, Paweł Gmerek, Michał Kuliński}
\date{November 2021}

\begin{document}

\maketitle
\section{Abstract}
Necklace is a tiny, imperative, statically, strongly typed language with Elixir-like syntax.

\section{Example syntax}

\begin{verbatim}
function do_stuff(a: int, b: int) -> int do
    return 2 + 2;
end

function complex -> int do
    a: int;
    b: int;
    c: int;
    d: int;
    a = 1;
    b = 2;
    c = 3;
    d = a + b + c;
    return d;
end

function array_operations(a: *int, size: int) -> void do
    i: int;
    for (i = 0; i < size; i += 1) {
        (a + i)* +=1
    }
end


function main do
    do_stuff(1, 2);
end
\end{verbatim}

\section{Tokens}
\subsubsection{Regular Expressions}
\begin{verbatim}
    @keywords = "(function|if|else|for|while|return|break|continue|->|do|end)"
    @varId = "[A-Za-z][A-Za-z0-9_]*"
    @int_lit = "([0-9])+" 
    @bool_lit = "true|false"
    @operator = "(\+|\-|\*|\/|%|<|>|>=|<=|==|!=|&&|\|\||!|=)"
    @comment = "~~.*"
    @special = "[\(\)\,\;\:\[\]\{\}]"
    @whitechar = "[\t\n\r\v\f\ ]"
    @type = int|bool
\end{verbatim}

\section{Grammar}
\begin{align*}
<start> &\longrightarrow \ <function>^* \\
<type> &\longrightarrow \ \textbf{bool} \ | \ \textbf{int} \ | \ \textbf{[}<type>\textbf{]} \ | \ \textbf{*} <type> \\
<return\_type> &\longrightarrow \ <type> \ | \ \textbf{void} \\
<function>  &\longrightarrow \\ 
    &| \ \textbf{function} <name> <arguments> \textbf{-\textgreater} <return\_type> \textbf{do} \ <function\_body> \ \textbf{end} \\
    &| \ \textbf{function} <name> \textbf{-\textgreater} <return\_type> \textbf{do} \ <function\_body> \ \textbf{end} \\
<arguments> &\longrightarrow \ \textbf{(} <function\_args>\textbf{)} \\
<function\_args> &\longrightarrow \ <name> \textbf{:} <type> (\textbf{,} <name> \textbf{:} <type>)^* \\
<function\_body> &\longrightarrow \ <declaration>^* <statement>^+ \\
<body> &\longrightarrow \ <statement>^* \\
<statement> &\longrightarrow \ <function\_call>; \\
    &| \ <name> \textbf{=} <expression> \textbf{;}\\ 
    &| \ \textbf{if } <expr> \textbf{do} <body> \textbf{else} <body>  \ \textbf{end} \\
    &| \ \textbf{for (} <expr> \textbf{,} <expr> \textbf{,} <expr> \textbf{) do} <body> \textbf{end} \\ 
    &| \ \textbf{while} <expr> \textbf{do} <body> \textbf{end} \\
    &| \ \textbf{return} <expr> \textbf{;} \\
    &| \ \textbf{return;} \\
    &| \ \textbf{break;} \\ 
    &| \ \textbf{continue;} \\ 
<declaration>  &\longrightarrow \ <name> \textbf{:} <type> \textbf{;} \\
<expression> &\longrightarrow \ <expression> <binary\_operator> <expression> \\
    &| \ <unary\_operator> <expression> \\
    &| \ ( <expression> ) \\
    &| \ <function\_call> \\
    &| \ <literal> \\
    &| \ <expression> \textbf{[} <expression> \textbf{]} \\
<unary\_operator> &\longrightarrow \ \textbf{*} \ | \ \textbf{-} \ | \ \textbf{!} \\
<binary\_operator> &\longrightarrow \ <arithmetic\_operator> \\ 
    &| \ <relational\_operator> \\
    &| \ <equality\_operator
    &| \ <conditional\_operator> \\
<arithmetic\_operator> &\longrightarrow \ \textbf{+} \
    | \ \textbf{-} \
    | \ \textbf{*} \
    | \ \textbf{/} \
    | \ \textbf{\%} \\
<relational\_operator> &\longrightarrow \ \textbf{\textless} \
    | \ \textbf{\textgreater} \
    | \ \textbf{\textless=} \
    | \ \textbf{\textgreater=} \\
<equality\_operator> &\longrightarrow \ \textbf{==} \ | \ \textbf{!=} \\
<conditional\_operator> &\longrightarrow \ \textbf{\&\&} \ | \ \textbf{$\mid\mid$} \\
<function\_call> &\longrightarrow \ <function\_name> ( <expr>^* ) \\
<function\_name> &\longrightarrow \ <name> \\
<literal> &\longrightarrow \ <int\_literal> \ | \ <bool\_literal> | <array\_literal> \\
<int\_literal> &\longrightarrow \ - <digit>^+|<digit>^+ \\
<bool\_literal> &\longrightarrow \ \textbf{true} \ | \ \textbf{false} \\
<array\_literal> &\longrightarrow \ [<expr>^*] \\
<identifier> &\longrightarrow \ <letter> \ | \ <identifier> <letter> \ | \ <identifier><digit> \\
\end{align*}
\subsection{Operators Precedence}
\begin{center}
\begin{tabular}{||c | c | c |  c|}
 \hline
 Priority & Category & Operator & Associativity\\
 \hline
 1 & Postfix  & $[$ $]$ & Left to right \\  
 \hline
 2 & Unary & $-$, $*$, $!$ & Right to left    \\
  \hline
 3 & Multiplicative & $*$, $/$, $\%$ & Left to Right    \\
 \hline
 4 & Additive & $+$, $-$ & Left to right \\ 
 \hline
 5 & Relational	& $<$, $>$, $<=$, $>=$ & Left to right\\
 \hline
 6 & Equality & $==$, $!=$ & Left to right \\
 \hline
 7 & Logical AND & $\&\&$ & Left to right \\
 \hline
 8 & Logical OR & $||$ & Left to right \\
 \hline
 9 & Assignment & $=$ & Right to left \\
 \hline
\end{tabular}
\end{center}
\section{Type system}
Necklace is a strongly typed language, so all type conversions have to be explicit.
\subsection{Base types}
\subsubsection{Boolean}
Declaration 
\begin{verbatim}
variable: bool;
\end{verbatim}
$$
\{true, false\}
$$
Corresponds to LLVMs i1 \url{https://releases.llvm.org/9.0.0/docs/LangRef.html#integer-type}
\subsubsection{Int}
Declaration 
\begin{verbatim}
variable: int;
\end{verbatim}

a 32 bit singed integer type \\
Corresponds to LLVMs i32 \url{https://releases.llvm.org/9.0.0/docs/LangRef.html#integer-type}
\subsubsection{Pointer}
\begin{verbatim}
variable: *<type>;
\end{verbatim}
Represents the location in memory of a variable
Corresponds to LLVMs pointer type \url{https://releases.llvm.org/9.0.0/docs/LangRef.html#pointer-type}
\subsubsection{Array}
Declaration 
\begin{verbatim}
variable: [<type>];
\end{verbatim}
Represents an array of variables of specified type
Corresponds to LLVMs array type \url{https://releases.llvm.org/9.0.0/docs/LangRef.html#array-type}

\subsection{Type inference}


\subsubsection{'-' unary operator}
$$
(+): int \longrightarrow int
$$

\subsubsection{'!' unary operator}
$$
(!): bool \longrightarrow bool
$$

\subsubsection{'*' unary operator}
$$
(*): pointer <type> \longrightarrow <type> 
$$

\subsubsection{'+' binary operator}
$$
(+): int \times int \longrightarrow int
$$

\subsubsection{'-' binary operator}
$$
(-): int \times int \longrightarrow int
$$

\subsubsection{'*' binary operator}
$$
(*): int \times int \longrightarrow int
$$

\subsubsection{'/' binary operator}
$$
(/): int \times int \longrightarrow int
$$

\subsubsection{'/' binary operator}
$$
(-): int \times int \longrightarrow int
$$
\subsubsection{'\%' binary operator}
$$
(\%): int \times int \longrightarrow int
$$
With behaviour defined as
$$
x\%y = r \ \ \ \ r = x - yk, x \in C
$$

\subsubsection{'toBool' conversion}
$$
toBool: int \longrightarrow bool
$$
With behaviour defined as
$$
toBool(x) =  \left\{ \begin{array}{ll}
false & x == 0  \\
true & otherwise \\
\end{array} \right.
$$

\subsubsection{'toInt' conversion}
$$
toInt: bool \longrightarrow int
$$
With behaviour defined as
$$
toInt(x) =  \left\{ \begin{array}{ll}
0 & x == false  \\
1 & x == true \\
\end{array} \right.
$$
\subsubsection{'$==$' binary operator}
$$
(==): int \times int \longrightarrow bool
$$
$$
(==): bool \times bool \longrightarrow bool
$$
\subsubsection{'$!=$' binary operator}
$$
(!=): int \times int \longrightarrow bool
$$
$$
(!=): bool \times bool \longrightarrow bool
$$

\subsubsection{'$<$' binary operator}
$$
(<): int \times int \longrightarrow bool
$$

\subsubsection{'$>$' binary operator}
$$
(>): int \times int \longrightarrow bool
$$
\subsubsection{'$<=$' binary operator}
$$
(<=): int \times int \longrightarrow bool
$$
\subsubsection{'$>=$' binary operator}
$$
(>=): int \times int \longrightarrow bool
$$
\subsubsection{'\&\&' binary operator}
$$
(\&\&): bool \times bool \longrightarrow bool
$$
\subsubsection{'$\mid\mid$' binary operator}
$$
(\mid\mid): bool \times bool \longrightarrow bool
$$
\subsubsection{'if' conditional operator}
$$
if \ bool \ do \ <block> \ end
$$
\subsubsection{'while' binary operator}
$$
while \ bool \ do \ <block> \ end
$$
\subsubsection{'for' binary operator}
$$
for \ <type> \ bool \ <type> \ do \ <block> \ end
$$

\section{Compiler architecture}
\subsection{Overview}

\subsection{Lexer}
The lexer is generated using the alex library for Haskell, which provides similar interface as lex. 

\subsection{Parser}
The parser is generated using the happy library for Haskell, which provides similar interface as yacc. 


\subsection{Semantic checker}
The language is statically and strongly checked. The compiler will perform a semantic analysis and throw errors if any of the types are not matching. 
TODO
1. Validate expressions and operator types
2. validate array literals are singular type
3. validate assignments have correct type
4. validate if all variables are declared



\subsection{Code generation}
For the generation of LLVM IR representation we use the llvm-hs library which provides bindings simplifying the LLVM code generation





\section{References}
\begin{enumerate}
    \item  Engineering a Compiler, by Keith D. Cooper & Linda Troczon
    \item MiT compilers course, decaf lang
\end{enumerate}

\end{document}
