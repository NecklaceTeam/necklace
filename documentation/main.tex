\documentclass{article}
\usepackage[utf8]{inputenc}
\usepackage{amsmath}
\usepackage{hyperref}
\usepackage{array}
\usepackage{comment}
\allowdisplaybreaks
\usepackage{graphicx}
\graphicspath{ {./images/} }

\title{Necklace}
\author{Maksymilian Demitraszek, Paweł Gmerek, Michał Kuliński}
\date{November 2021}

\begin{document}

\maketitle
\section{Abstract}
Necklace is a tiny, imperative, statically, strongly typed language with Elixir-like syntax.
Made for Group Project 1 \& 2 at Jagiellonian University. Source code is hosted on GitHub repository: \url{https://github.com/NecklaceTeam/necklace}

\section{Roles}
\begin{itemize}
    \item Engineering and Language Design: \textit{Maksymilian Demitraszek, Paweł Gmerek}
    \item Testing: \textit{Michał Kuliński}
\end{itemize}
\section{Compiler architecture}
\subsection{Overview}
Necklace is a 2 phase compiler
\begin{itemize}
    \item Frontend - Lexer, Parser, Semantic checker
    \item Backend - code generation to LLVM
\end{itemize}
\begin{center}
    \includegraphics[scale=0.5]{compiler}
\end{center}
\subsection{Lexer}
The lexer is generated using the alex library for Haskell, which provides similar interface as lex. 

\subsection{Parser}
The parser is generated using the happy library for Haskell, which provides similar interface as yacc. 


\subsection{Semantic checker}
The language is statically and strongly checked. The compiler will perform a semantic analysis and throw errors if any of the types are not matching. 



\subsection{Code generation}
For the generation of LLVM IR representation we use the llvm-hs library which provides bindings simplifying the LLVM code generation



\section{Example syntax}
Every Necklacke program needs to have a main function that returns an integer.
\begin{verbatim}
function do_stuff(a: int, b: int) -> int do
    return 2 + 2;
end

function complex -> int do
    a: int;
    b: int;
    c: int;
    d: int;
    a = 1;
    b = 2;
    c = 3;
    d = a + b + c;
    return d;
end

function array_operations(a: *int, size: int) -> void do
    i: int;
    for (i = 0; i < size; i += 1) {
        (a >> i)* +=1
    }
end


function main -> int do
    do_stuff(1, 2);
    return 0;
end
\end{verbatim}

\section{Tokens}
\subsubsection{Regular Expressions}
\begin{verbatim}
    @keywords = "(function|if|else|while|return|break|continue|->|do|end|alloc|free)"
    @varId = "[A-Za-z][A-Za-z0-9_]*"
    @int_lit = "([0-9])+" 
    @bool_lit = "true|false"
    @operator = "(\+|\-|\*|\/|%|<|>|>=|<=|==|!=|&&|\|\||!|=|>>|<<)"
    @comment = "~~.*"
    @special = "[\(\)\,\;\:\[\]\{\}]"
    @whitechar = "[\t\n\r\v\f\ ]"
    @type = int|bool
\end{verbatim}

\section{Grammar}
\begin{align*}
<start> &\longrightarrow \ <function>^* \\
<type> &\longrightarrow \ \textbf{bool} \ | \ \textbf{int} \ | \ \textbf{*} <type> \ | \textbf{[} <type> \textbf{]}\\
<return\_type> &\longrightarrow \ <type> \ | \ \textbf{void} \\
<function>  &\longrightarrow  \textbf{function} <name> <function\_type> \textbf{do} \ <function\_body> \ \textbf{end} \\
<function\_type> &\longrightarrow \ 
\textbf{(} <function\_args> \textbf{)} \ 
\textbf{-\textgreater} <return\_type>\\
&| \ \textbf{(} <function\_args> \textbf{)} \\
&| \ \textbf{()} \  \textbf{-\textgreater} <return\_type> \\
&| \ \textbf{-\textgreater} <return\_type> \\
&| \ \textbf{(} <function\_args> \textbf{)} \\
&| \  \textbf{()} \\
&| \ \epsilon \\
<function\_args> &\longrightarrow \  <function\_args> \textbf{,} <declaration> | <declaration> \\
<function\_body> &\longrightarrow \ <declaration>^* <statement>^+ \\
<body> &\longrightarrow \ <statement>^* \\
<statement> &\longrightarrow  \ <name> \textbf{=} <expression> \textbf{;}\\ 
    &| \ \textbf{if } <expr> \textbf{do} <body> \textbf{else} <body>  \ \textbf{end} \\
    &| \ \textbf{for (} <expr> \textbf{,} <expr> \textbf{,} <expr> \textbf{) do} <body> \textbf{end} \\ 
    &| \ \textbf{while} <expr> \textbf{do} <body> \textbf{end} \\
    &| \ <expr>; \\
    &| \ \textbf{free} <expr>; \\
    &| \ \textbf{return} <expr> \textbf{;} \\
    &| \ \textbf{return;} \\
<declaration>  &\longrightarrow \ <name> \textbf{:} <type> \textbf{;} \\
<expression> &\longrightarrow \ <operator>\\
    &| \ \textit{@varId} \textbf{(} <expression>^* \textbf{)} \\
    &| \ \textbf{(} <expression> \textbf{)} \\
    &| \ <operator> \\
    &| \ <literal> \\
    &| \ \textit{@varId} \\
<operator> &\longrightarrow \ <expression> \textbf{*} <expression \\
    &| \ <expression> \textbf{-} <expressiosn> \\
    &| \ <expression> \textbf{+} <expression> \\
    &| \ <expression> \textbf{/} <expression> \\
    &| \ <expression> \textbf{\%} <expression> \\
    &| \ <expression> \textbf{$<$} <expression> \\
    &| \ <expression> \textbf{$<=$} <expression> \\
    &| \ <expression> \textbf{$>$} <expression> \\
    &| \ <expression> \textbf{$>=$} <expression> \\
    &| \ <expression> \textbf{==} <expression> \\
    &| \ <expression> \textbf{!=} <expression> \\
    &| \ <expression> \textbf{\&\& } <expression> \\
    &| \ <expression> \textbf{$\Vert$} <expression> \\
    &| \ <expression> \textbf{$>>$} <expression> \\
    &| \ <expression> \textbf{$<<$} <expression> \\
    &| \ \textbf{!} <expression> \\
    &| \ \textbf{*} <expression> \\
    &| \ \textbf{-} <expression> \\
    &| \ <expression> \textbf{=} <expression> \\
    &| \ <expression> \textbf{[} <expression> \textbf{]} \\
    &| \ \textbf{alloc} <allocable> \\
<allocable> &\longrightarrow \ <type> \textbf{[} <expression> \textbf{]} \\
<literal> &\longrightarrow \ <@int\_lit> \ | \ <@bool\_lit> \\
\end{align*}
\subsection{Operators Precedence}
\begin{center}
\begin{tabular}{||c | c | c |  c|}
 \hline
 Priority & Category & Operator & Associativity\\
 \hline
 1 & Postfix  & $[$ $]$ & Left to right \\  
 \hline
 2 & Unary & $-$, $*$, $!$ & Right to left    \\
  \hline
 3 & Multiplicative & $*$, $/$, $\%$ & Left to Right    \\
 \hline
 4 & Additive & $+$, $-$ & Left to right \\ 
 \hline
 5 & Relational	& $<$, $>$, $<=$, $>=$ & Left to right\\
 \hline
 6 & Equality & $==$, $!=$ & Left to right \\
 \hline
 7 & Logical AND & $\&\&$ & Left to right \\
 \hline
 8 & Logical OR & $||$ & Left to right \\
 \hline
 9 & Pointer move & $>>$, $<<$ & Left to right \\
 \hline
 10 & Assignment & $=$ & Right to left \\
 \hline
\end{tabular}
\end{center}
\section{Type system}
Necklace is a strongly typed language, so all type conversions have to be explicit. In fact they are not allowed at all.
\subsection{Base types}
\subsubsection{Boolean}
Declaration 
\begin{verbatim}
variable: bool;
\end{verbatim}
$$
\{true, false\}
$$
Corresponds to LLVMs i1 \url{https://releases.llvm.org/9.0.0/docs/LangRef.html#integer-type}
\subsubsection{Int}
Declaration 
\begin{verbatim}
variable: int;
\end{verbatim}

a 32 bit singed integer type \\
Corresponds to LLVMs i32 \url{https://releases.llvm.org/9.0.0/docs/LangRef.html#integer-type}
\subsubsection{Pointer}
\begin{verbatim}
variable: *<type>;
\end{verbatim}
Represents the location in memory of a variable
Corresponds to LLVMs pointer type \url{https://releases.llvm.org/9.0.0/docs/LangRef.html#pointer-type}
\subsubsection{Array}
Declaration 
\begin{verbatim}
variable: [<type>];
\end{verbatim}
Represents an array of variables of specified type
Corresponds to LLVMs array type \url{https://releases.llvm.org/9.0.0/docs/LangRef.html#array-type}

\subsection{Type inference}


\subsubsection{'-' unary operator}
$$
(+): int \longrightarrow int
$$

\subsubsection{'!' unary operator}
$$
(!): bool \longrightarrow bool
$$

\subsubsection{'*' unary operator}
$$
(*): pointer <type> \longrightarrow <type> 
$$

\subsubsection{'+' binary operator}
$$
(+): int \times int \longrightarrow int
$$

\subsubsection{'-' binary operator}
$$
(-): int \times int \longrightarrow int
$$

\subsubsection{'*' binary operator}
$$
(*): int \times int \longrightarrow int
$$

\subsubsection{'/' binary operator}
$$
(/): int \times int \longrightarrow int
$$

\subsubsection{'/' binary operator}
$$
(-): int \times int \longrightarrow int
$$
\subsubsection{'\%' binary operator}
$$
(\%): int \times int \longrightarrow int
$$
\begin{comment}
    \subsubsection{'toBool' conversion}
    $$
    toBool: int \longrightarrow bool
    $$
    With behaviour defined as
    $$
    toBool(x) =  \left\{ \begin{array}{ll}
    false & x == 0  \\
    true & otherwise \\
    \end{array} \right.
    $$
    
    \subsubsection{'toInt' conversion}
    $$
    toInt: bool \longrightarrow int
    $$
    With behaviour defined as
    $$
    toInt(x) =  \left\{ \begin{array}{ll}
    0 & x == false  \\
    1 & x == true \\
    \end{array} \right.
    $$
\end{comment}

\subsubsection{'$==$' binary operator}
$$
(==): int \times int \longrightarrow bool
$$
$$
(==): bool \times bool \longrightarrow bool
$$
\subsubsection{'$!=$' binary operator}
$$
(!=): int \times int \longrightarrow bool
$$
$$
(!=): bool \times bool \longrightarrow bool
$$

\subsubsection{'$<$' binary operator}
$$
(<): int \times int \longrightarrow bool
$$

\subsubsection{'$>$' binary operator}
$$
(>): int \times int \longrightarrow bool
$$
\subsubsection{'$<=$' binary operator}
$$
(<=): int \times int \longrightarrow bool
$$
\subsubsection{'$>=$' binary operator}
$$
(>=): int \times int \longrightarrow bool
$$
\subsubsection{'\&\&' binary operator}
$$
(\&\&): bool \times bool \longrightarrow bool
$$
\subsubsection{'$\mid\mid$' binary operator}
$$
(\mid\mid): bool \times bool \longrightarrow bool
$$
\subsubsection{'if' conditional operator}
$$
if \ bool \ do \ <block> \ else <block> end
$$
\subsubsection{'while' binary operator}
$$
while \ bool \ do \ <block> \ end
$$
\section{Tests}
Necklace test suite includes unit tests for lexer and context sensitive analysis and integration tests for basic language functionalities.
\section{Installation guide}
\subsection{Requirements}
\begin{itemize}
    \item llvm9 on your linux machine
    \item If llvm-hs does not detect llvm-config-9 you can symlink it as llvm-conifg and add it to the PATH
\end{itemize}
\subsection{How to start}
You need installed llvm9 and stack
\begin{verbatim}
$ stack install alex happy
$ stack install
$ stack run necklace-exe `<file_name>.nck`
\end{verbatim}

\section{Results}
We managed to implement an insertion sort algorithm in Necklace
\begin{verbatim}
function insertionSort(arr: [int], size: int) -> [int] do
  i: int;
  j: int;
  key: int;

  if (2 > size) do
    return arr;
  else
    i = 1;
    while (i < size) do
      key = arr[i];
      j = i - 1;
      while (j >= 0 && arr[j] > key) do
        arr[j+1] = arr[j];
        j = j-1;
      end
      arr[j+1] = key;
      i = i + 1;
    end
    return arr;
  end
  return arr;
end


function main -> int do
    c: [int];
    i: int;
    k: int;
    i = 0;
    k = 10;

    c = alloc int[10];
    while (i < 10) do
      c[i] = k;
      printInt(c[i]);
      k = k - 1;
      i = i + 1;
    end
    
    insertionSort(c, 10);
    
    i = 0;

    while (i < 10) do
      printInt(c[i]);
      i = i + 1;
    end
    free c;
    return 0;
end

\end{verbatim}
When compiled and run, it returns to stdout an array of numbers from 1 to 10 in reverse order, sorts them and prints it sorted.
\begin{verbatim}
    10
    9
    8
    7
    6
    5
    4
    3
    2
    1
    1
    2
    3
    4
    5
    6
    7
    8
    9
    10
\end{verbatim}

\section{Libraries}
\begin{enumerate}
    \item \url{https://github.com/simonmichael/shelltestrunner}
    \item https://github.com/haskell/alex
    \item https://github.com/haskell/happy
\end{enumerate}


\section{Bibliography}
\begin{enumerate}
    \item  Engineering a Compiler 2nd edition, by Keith D. Cooper & Linda Troczon
    \item MIT Decaf language
    \url{https://ocw.mit.edu/courses/6-035-computer-language-engineering-spring-2010/1b21d612e072d1aad797f040a8bfdb6b_MIT6_035S10_decaf.pdf}
    \item Write you a Haskell 2 Lexer \& Parser \url{https://jktkops.github.io/Write-You-a-Haskell-2/}
    \item ANSI C grammar \url{https://www.lysator.liu.se/c/ANSI-C-grammar-y.html}
    \item Micro C blogposts
    \url{https://blog.josephmorag.com/tags/llvm/}
\end{enumerate}

\end{document}
